% !TeX root = ../main.tex
\documentclass[class=article]{standalone}

\begin{document} 
\centerline{\Huge \bf Question 4}

\centerline{Analyse de \lstinline{construire_noeud}}

Notons que cette fonction possède un fonction privée récursive (\lstinline{build_node}) 
et que la seule appel de la fonction \lstinline{construire_noeud}
est à appel à cette fonction privée. Nous évalurons donc la fonction \lstinline{build_node}.

\section*{Taille de l'instance}

La taille de l'instance est $n$, le nombre d'éléments compris entre
l'index \lstinline{beginIndex} et l'index \lstinline{endIndex} 
inclusivement. ($n = $\lstinline{endIndex}$-$\lstinline{beginIndex}$+1$)

\section*{Opération de base}

Nous allons choisir comme opération de base l'appel à la comparaison \lstinline{nbPoints == 1}

Cet opération est l'opération la plus effectué dans la fonction, car il n'y a pas de boucle 
itérative suivant cet opération et qu'elle n'est pas imbriqué dans des conditionnels.

Le nombre d'opération est dépendant uniquement dans la valeur de $n$.

\section*{Analyse de la récusion}

La récurence peut être donnée par la fonction suivante

\[
  C(n) =
  \begin{cases}
    1 & \text{si } n \leq 1 \\
    1 + C(\ceil{\frac{n}{2}}) + C(\floor{\frac{n}{2}})  & \text{si } n > 1 \\
  \end{cases}
\]

Nous la modifirons ainsi:

\begin{deriv}
  C\pars{n} =
  \begin{cases}
    1 & \text{si } n \leq 1 \\
    1 + C(\ceil{\frac{n}{2}}) + C(\floor{\frac{n}{2}})  & \text{si } n > 1 \\
  \end{cases}
  \<\Rightarrow
  \commentaire{En supposant que $n = 2^k$}
  C\pars{2^k} =
  \begin{cases}
    1 & \text{si } k = 0 \\
    1 + C(\ceil{\frac{2^k}{2}}) + C(\floor{\frac{2^k}{2}})  & \text{si } k \geq 1 \\
  \end{cases}
  \<\Rightarrow
  \commentaire{Puisque $\frac{2^k}{2}$ est toujours un entier}
  C\pars{2^k} =
  \begin{cases}
    1 & \text{si } k = 0 \\
    1 + 2 \cdot C(\frac{2^k}{2})  & \text{si } k \geq 1 \\
  \end{cases}
  \<\Rightarrow
  \commentaire{Rammener sur $n$}
  C\pars{n} =
  \begin{cases}
    1 & \text{si } n \leq 1 \\
    1 + 2 \cdot C(\frac{n}{2})  & \text{si } n > 1 \wedge n = 2^k \forall k \in \Natural^+ \\
  \end{cases}
\end{deriv}

Nous avons ainsi une forme où le théorème général avec

$r = 2 \wedge b = 2 \wedge f\pars{n} = n^0 \wedge d = 0$

Nous obtenons la forme 3 du théorème général

$r = 2 > 1 = 2^0 = b^d$

donc

$C\pars{n} \in \Theta\pars{n^{log_2 2}} \equiv C\pars{n} \in \Theta\pars{n} $

\section*{Conclusion}

La fonction \lstinline{construire_noeud} s'exécute donc en temps linéaire par rapport au nombre de points.

\end{document}