% !TeX root = ../main.tex
\documentclass[class=article]{standalone}

\begin{document} 
\centerline{\Huge \bf Question 4}

\centerline{Analyse de \lstinline{construire_noeud}}

Notons que cette fonction possède une fonction privée récursive (\lstinline{build_node}) 
et que la seule opération de \lstinline{construire_noeud}
est l'appel à cette fonction privée. Nous évaluons donc la fonction \lstinline{build_node}.

\section*{Taille de l'instance}

La taille de l'instance est $n$, le nombre d'éléments compris entre
l'index \lstinline{beginIndex} et l'index \lstinline{endIndex} 
inclusivement. ($n = $\lstinline{endIndex}$-$\lstinline{beginIndex}$+1$)

\section*{Opération de base}

Nous devons choisir comme opération de base l'appel à la fonction \lstinline{fusion}.

Cette opération est l'opération la plus effectuée dans la fonction à une constante prêt, car il n'y a pas de boucles 
itératives suivant cette opération et qu'elle est imbriquée dans une conditionnels mais dont il n'y a pas d'autre traitement autour.

Le nombre d'opérations dépend uniquement dans la valeur de $n$.

\section*{Analyse de la récursion}

La récurrence peut être donnée par la fonction suivante

\[
  C(n) =
  \begin{cases}
    0 & \text{si } n \leq 1 \\
    C\pars{\ceil{\frac{n}{2}}} + C\pars{\floor{\frac{n}{2}}} + C_{\lstinline{fusion}}\pars{n}  & \text{si } n > 1 \\
  \end{cases}
\]

Nous devons donc commencer par analyser la méthode de $C_{\lstinline{fusion}}\pars{n}$.

\subsubsection*{Analyse de \lstinline{fusion}}
\subsubsubsection*{Taille de l'instance}

La taille de l'instance est $n$, le nombre de feuille dans l'arbre du noeud, 
soit la somme des feuilles de son noeud gauche et de son noeud droit.

\subsubsubsection*{Opération de base}

Remarquons que tous les opérations dans cette méthodes sont de complexité constante (ou constante ammorti).

Nous prendrons comme opération de base la comparaison \listinline{currentParentIndex < nbLeafsParent} de
la boucle \lstinline{for}.

Cette opération est la plus appelé de la méthode, puisqu'il n'y a qu'une seule boucle et que c'est l'opérateur de sortie de cette boucle.

Il n'y a pas de meilleur ou de pire cas, le nombre de fois qu'est appelé cette opération peut être donné par la sommation suivante:

$C_{\lstinline{fusion}}\pars{n} = \sum\limits_{i=0}^{n-1}1$

Résolvons cette sommation

\begin{deriv}
  C_{\lstinline{fusion}}\pars{n}
  \<= 
  \commentaire{Définition de \lstinline{fusion}}
  \sum\limits_{i=0}^{n-1}1
  \<=
  \commentaire{Sommation d'une constante}
  \pars{n-1 - 0 + 1} \cdot 1
  \<=
  \commentaire{Simplification}
  n
\end{deriv} 

Donc C_{\lstinline{fusion}}\pars{n} \in \Theta\pars{n}

\subsection*{Retour à l'analyse de \lstinline{construire_noeud}}
Nous modifions la récurrence ainsi:

\begin{deriv}
  C\pars{n} =
  \begin{cases}
    0 & \text{si } n \leq 1 \\
    C\pars{\ceil{\frac{n}{2}}} + C\pars{\floor{\frac{n}{2}}} + C_{\lstinline{fusion}}\pars{n}  & \text{si } n > 1 \\
  \end{cases}
  \<\Rightarrow
  \commentaire{En supposant que $n = 2^k$}
  C\pars{2^k} =
  \begin{cases}
    0 & \text{si } k = 0 \\
    C\pars{\ceil{\frac{2^k}{2}}} + C\pars{\floor{\frac{2^k}{2}}} + C_{\lstinline{fusion}}\pars{2^k}  & \text{si } k \geq 1 \\
  \end{cases}
  \<\Rightarrow
  \commentaire{Puisque $\frac{2^k}{2}$ est toujours un entier}
  C\pars{2^k} =
  \begin{cases}
    1 & \text{si } k = 0 \\
    1 + 2 \cdot C(\frac{2^k}{2})  & \text{si } k \geq 1 \\
  \end{cases}
  \<\Rightarrow
  \commentaire{Ramener sur $n$}
  C\pars{n} =
  \begin{cases}
    1 & \text{si } n \leq 1 \\
    1 + 2 \cdot C(\frac{n}{2})  & \text{si } n > 1 \wedge n = 2^k \forall k \in \Natural^+ \\
  \end{cases}
\end{deriv}

Nous avons ainsi une forme où le théorème général avec

$r = 2 \wedge b = 2 \wedge f\pars{n} = n^0 \wedge d = 0$

Nous obtenons la forme 3 du théorème général

$r = 2 > 1 = 2^0 = b^d$

donc

$C\pars{n} \in \Theta\pars{n^{log_2 2}} \equiv C\pars{n} \in \Theta\pars{n} $

\section*{Conclusion}

La fonction \lstinline{construire_noeud} s'exécute donc en temps linéaire par rapport au nombre de points.

\end{document}