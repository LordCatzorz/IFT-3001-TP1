% !TeX root = ../main.tex
\documentclass[class=article]{standalone}

\begin{document} 
\centerline{\Huge \bf Question 3}
\section*{Algorithme 1}

Nous allons séparer l'analyse de cet algorithme en deux blocs distincts. 
Le bloc A portera sur les boucles \lstinline{for} imbriquées.
Le bloc B portera sur la dernière boucle \lstinline{for}.
Les opérations restantes sont des opérations atomiques et ne changeront donc pas l'analyse asymptotique.

\subsection*{Bloc A}
\subsubsection*{Taille de l'instance}
La taille de l'instance est $n$, la valeur fournie en paramètre.

\subsubsection*{Opérateur de base}
Nous allons prendre comme opérateur de base l'addition de c avec 1 (\lstinline{c+1}).

Le nombre de fois que cet opérateur est atteint dépend uniquement de la valeur de $n$.

Le nombre de fois que l'opérateur de base est appelé peut être donné par la sommation suivante:

$C_A\pars{n} = \sum\limits_{i=1}^n \sum\limits_{j=i^3}^{n^3}1$

Résolvons cette équation.

\begin{deriv}
    C_A\pars{n}
    \<=
    \commentaire{Définition de $C_A\pars{n}$}
    \sum\limits_{i=1}^n \sum\limits_{j=i^3}^{n^3}1
    \<=
    \commentaire{Sommation d'une constante}
    \sum\limits_{i=1}^n \pars{\pars{n^3 - i^3 + 1} \cdot 1}
    \<=
    \commentaire{Simplification}
    \sum\limits_{i=1}^n \pars{n^3 - i^3 + 1}
\end{deriv}


\subsubsection*{Analyse asymptotique}
Nous allons borner la sommation par une intégrale.

La sommation est non croissante.

Avec $f(i) = \sum\limits_{i=1}^n\pars{n^3 - i^3 + 1}$;

Avec $g(x) = \int\limits_1^{n+1} \pars{n^3 - x^3 + 1} dx$;

Avec $h(x) = \int\limits_0^{n} \pars{n^3 - x^3 + 1} dx$;

Résolvons $g(x) \leq f(i) \leq h(x)$, l'approximation par intégrale d'une sommation nulle part croissante:

Résolvons $g(x) = \int\limits_1^{n+1} \pars{n^3 - x^3 + 1} dx$,
soit une borne inférieure de $\sum\limits_{i=1}^n \pars{n^3 - i^3 + 1}$.

\begin{deriv}
    C_A\pars{n}
    \<=
    \commentaire{Définition de $C_A\pars{n}$}
    f(i)
    \<\geq
    \commentaire{Définition de $g(x)$}
    g(x)
    \<=
    \commentaire{Définition de $g(x)$}
    \int\limits_1^{n+1} \pars{n^3 - x^3 + 1} dx
    \<=
    \commentaire{Distribution de l'intégration}
    \int\limits_1^{n+1} n^3 dx
    - \int\limits_1^{n+1} x^3 dx
    + \int\limits_1^{n+1} 1 dx
    \<=
    \commentaire{Intégration}
    \intr{1}{n+1}{n^3x}
    - \intr{1}{n+1}{\frac{x^4}{4}} dx
    + \intr{1}{n+1}{x} dx
    \<=
    \commentaire{Évaluation}
    \pars{n^3(n+1) - n^3(1)}
    - \pars{\frac{\pars{n+1}^4}{4} - \frac{1^4}{4}}
    + \pars{n+1 - 1}
    \<=
    \commentaire{Simplification}
    \pars{n^4 +n^3 - n^3}
    - \pars{\frac{\pars{n+1}^4-1}{4}}
    + n
    \<=
    \commentaire{Simplification}
    n^4 - \frac{\pars{n^2+2n+1}^2-1}{4} + n
    \<=
    \commentaire{Simplification}
    n^4 - \frac{n^4+4n^3+6n^2+4n+1-1}{4} + n
    \<=
    \commentaire{Simplification}
    n^4 - \frac{n^4+4n^3+6n^2+4n}{4} + n
    \<=
    \commentaire{Mettre sur la même base}
    \frac{4n^4}{4} - \frac{n^4+4n^3+6n^2+4n}{4} + \frac{4n}{4}
    \<=
    \commentaire{Ramener sur la même division}
    \frac{4n^4 - \pars{n^4+4n^3+6n^2+4n} + 4n}{4}
    \<=
    \commentaire{Distribuer la soustraction}
    \frac{4n^4 - n^4 - 4n^3 - 6n^2 - 4n + 4n}{4}
    \<=
    \commentaire{Simplifier}
    \frac{3n^4 - 4n^3 - 6n^2}{4}
    \<=
    \commentaire{Extraction de $\frac{n^2}{4}$}
    \frac{n^2}{4}\pars{3n^2 - 4n - 6}
    \<\geq
    \commentaire{$\forall n \geq 6$}
    \frac{n^2}{4}\pars{3n^2 - 4n - n}
    \<=
    \commentaire{Simplification}
    \frac{n^2}{4}\pars{3n^2 - 3n}
    \<=
    \commentaire{Simplification}
    \frac{n^3}{4}\pars{3n - 3}
    \<\geq
    \commentaire{$\forall n \geq 3$}
    \frac{n^3}{4}\pars{3n - n}
    \<=
    \commentaire{Simplification}
    \frac{n^3}{4}\pars{2n}
    \<=
    \commentaire{Simplification}
    \frac{2n^4}{4}
    \<=
    \commentaire{Simplification}
    \frac{n^4}{2}
    \<\in
    \commentaire{Définition de $\Omega$}
    \Omega\pars{n^4}
\end{deriv}


Résolvons $h(x) = \int\limits_0^{n} \pars{n^3 - x^3 + 1} dx$,
soit une borne supérieure de $\sum\limits_{i=1}^n \pars{n^3 - i^3 + 1}$.

\begin{deriv}
    C_A\pars{n}
    \<=
    \commentaire{Définition de $C_A\pars{n}$}
    f(i)
    \<\leq
    \commentaire{Définition de $h(x)$}
    h(x)
    \<=
    \commentaire{Définition de $h(x)$}
    \int\limits_0^{n} \pars{n^3 - x^3 + 1} dx
    \<=
    \commentaire{Distribution de l'intégration}
    \int\limits_0^{n} n^3 dx
    - \int\limits_0^{n} x^3 dx
    + \int\limits_0^{n} 1 dx
    \<=
    \commentaire{Intégration}
    \intr{0}{n}{n^3x}
    - \intr{0}{n}{\frac{x^4}{4}} dx
    + \intr{0}{n}{x} dx
    \<=
    \commentaire{Évaluation}
    \pars{n^3(n) - n^3(0)}
    - \pars{\frac{\pars{n}^4}{4} - \frac{0^4}{4}}
    + \pars{n - 0}
    \<=
    \commentaire{Simplification}
    n^4 - \frac{n^4}{4} + n
    \<=
    \commentaire{Simplification}
    \frac{3}{4}n^4 + n
    \<\leq
    \commentaire{$\forall n \geq 0$}
    \frac{3}{4}n^4 + n^4
    \<=
    \commentaire{Simplification}
    \frac{7}{4}n^4
    \<\in 
    \commentaire{Définition de $\BigO$}
    \BigO\pars{n^4}
\end{deriv}

Donc $C_A\pars{n} \in \Theta\pars{n^4}$, puisque $f(i) \in \Omega\pars{n^4}$ et $f(i) \in \BigO\pars{n^4}$


\subsection*{Bloc B}
\subsubsection*{Taille de l'instance}
La taille de l'instance est $n$, la valeur fournie en paramètre.

\subsubsection*{Opérateur de base}
Nous allons prendre comme opérateur de base l'addition de c avec 1 (\lstinline{c+1}).

Le nombre de fois que cet opérateur est atteint dépend uniquement de la valeur de $n$.

Le nombre de fois que l'opérateur de base est appelé peut être donné par la sommation suivante:

$C_B\pars{n} = \sum\limits_{i=1}^{\floor{\sqrt{n}}} 1$

Résolvons cette équation.

\begin{deriv}
    C_B\pars{n}
    \<=
    \commentaire{Définition de $C_B\pars{n}$}
    \sum\limits_{i=1}^{\floor{\sqrt{n}}} 1
    \<=
    \commentaire{Sommation d'une constante}
    \pars{\pars{\floor{\sqrt{n}} - 1 + 1} \cdot 1}
    \<=
    \commentaire{Simplification}
    \floor{\sqrt{n}}
\end{deriv}

\subsubsection*{Analyse asymptotique}
En utilisant la définition de la fonction plancher, nous avons:

$\sqrt{n} - 1 < C_B\pars{n} \leq \sqrt{n}$

Utilisons ces inéquations pour trouver la notation asymptotique de $C_B\pars{n}$

Trouvons $\Omega$

\begin{deriv}
    C_B\pars{n}
    \<=
    \commentaire{Définition de $C_B\pars{n}$}
    \floor{\sqrt{n}}
    \<>
    \commentaire{Définition de la fonction plancher}
    \sqrt{n} - 1
    \<\geq
    \commentaire{$\forall n \geq 4$}
    \sqrt{n} - \frac{\sqrt{n}}{2}
    \<=
    \commentaire{Simplification}
    \frac{\sqrt{n}}{2}
    \<=
    \commentaire{Définition de la racine carrée}
    \frac{n^{\frac{1}{2}}}{2}
    \<\in
    \commentaire{Définition $\Omega$}
    \Omega\pars{n^{\frac{1}{2}}}
\end{deriv}

Trouvons $\BigO$
\begin{deriv}
    C_B\pars{n}
    \<=
    \commentaire{Définition de $C_B\pars{n}$}
    \floor{\sqrt{n}}
    \<\leq
    \commentaire{Définition de la fonction plancher}
    \sqrt{n}
    \<=
    \commentaire{Définition de la racine carrée}
    n^{\frac{1}{2}}
    \<\in
    \commentaire{Définition $\BigO$}
    \BigO\pars{n^{\frac{1}{2}}}
\end{deriv}

Donc $C_B\pars{n} \in \Theta\pars{n^{\frac{1}{2}}}$

\subsection*{Résultat final}

Nous avons pour le bloc A que $C_A\pars{n} \in \Theta\pars{n^4}$
et pour le bloc B que $C_B\pars{n} \in \Theta\pars{n^{\frac{1}{2}}}$.

Puisque chacun des blocs s'exécute une fois, nous pouvons conclure
$C\pars{n} \in \Theta\pars{n^4} + \Theta\pars{n^{\frac{1}{2}}}$.

En utilisant la règle du maximum, nous pouvons avoir:
$C\pars{n} \in \Theta\pars{n^4}$.

\pagebreak

\section*{Algorithme 2}
\subsection*{Taille de l'instance}
La taille de l'instance est $n$, le nombre d'éléments dans l'étendue, 
soit 1 plus la différence entre 
le paramètre $r$ (position d'un élément à droite) et 
le paramètre $l$ (position d'un élément à gauche).

$n = (r - l) + 1$

\subsubsection*{Opérateur de base}
L'opérateur de base à utiliser est la comparaison \lstinline{l < r} du premier \lstinline{if}.
C'est l'opération exécuter le plus souvent, car il n'y a aucune boucle dans la méthode et que
c'est la première opération effectuée.

\subsubsection*{Définition de la récurrence}

Le nombre $C\pars{n}$ d'opérations de base effectuées sur un vecteur est donné par la récurrence suivante:

\[
  C\pars{n} =
  \begin{cases}
    1 & \text{si } n \leq 2 \\
    1 + 3 \cdot C\pars{n-\floor{\frac{n}{3}}} & \text{si } n > 2 \\
  \end{cases}
\]

Explication de la récurrence:
\begin{enumerate}
    \item Si $n = 1+(l-r) \leq 1$, alors on effectue la comparaison (opérateur de base), 
    puis on sort de la méthode directement.
    \item Si $n = 1+(l-r) = 2$, alors on effectue la comparaison (opérateur de base), 
    puis on effectue une deuxième comparaison (des valeurs), en effectuant ou non 
    un échange, on n'entre pas par la suite dans dernier \lstinline{if}, puis on sort de la méthode.
    \item Si $n = l-r \geq 3$, alors on effectue la comparaison (opérateur de base),
    puis on effectue une deuxième comparaison (des valeurs), en effectuant ou non 
    un échange, on entre par la suite dans le dernier \lstinline{if}, ou l'on
    va appeler trois fois la récursion. Deux fois en retirant $k = \floor{\frac{n}{3}}$ 
    de la valeur de $r$ et une fois en l'ajoutant à $l$, ce qui a pour effet de 
    retirer retirer $k$ à $n$, d'où le $n-\floor{\frac{n}{3}}$.
\end{enumerate}

\subsubsection*{Résolution de la récurrence (pour notation asymptotique)}
Supposons $n = 3^k \equiv log_{3}\pars{n} = k$.

Nous pouvons redéfinir $C\pars{n}$ ainsi:

\begin{deriv}
    C\pars{n}
    \<= 
    \commentaire{Définition de $C\pars{n}$}
    1 + 3 \cdot C\pars{n-\floor{\frac{n}{3}}}
    \<= 
    \commentaire{Définition de $n$}
    1 + 3 \cdot C\pars{3^k-\floor{\frac{3^k}{3}}}
    \<= 
    \commentaire{Règle des exposants}
    1 + 3 \cdot C\pars{3^k-\floor{3^{k-1}}}
    \<= 
    \commentaire{Fonction plancher sur nombre entier}
    1 + 3 \cdot C\pars{3^k-3^{k-1}}
    \<= 
    \commentaire{Règle des exposants}
    1 + 3 \cdot C\pars{3\cdot3^{k-1}-3^{k-1}}
    \<= 
    \commentaire{Simplification ($3x-x = 2x$)}
    1 + 3 \cdot C\pars{2\cdot3^{k-1}}
    \<= 
    \commentaire{Définition de $k$}
    1 + 3 \cdot C\pars{2\cdot3^{log_3(n)-1}}
    \<= 
    \commentaire{Propriété des logs}
    1 + 3 \cdot C\pars{2\cdot\frac{3^{log_3(n)}}{3^1}}
    \<= 
    \commentaire{Simplification}
    1 + 3 \cdot C\pars{2\cdot\frac{n}{3}}
    \<= 
    \commentaire{Réécriture}
    1 + 3 \cdot C\pars{\frac{n}{\frac{3}{2}}}
    \<= 
    \commentaire{Réécriture}
    3 \cdot C\pars{\frac{n}{\frac{3}{2}}} + 1
    \<\in
    \commentaire{Théorème général avec $r=3$, $b=\frac{3}{2}$ et $d=0$\\
                 et $r = 3 > b^d = \pars{\frac{3}{2}}^0 = 1$\\
                 $d = 0$, puisque $1 \in \Theta\pars{n^0}$}
    \Theta\pars{n^{log_{\frac{3}{2}}3}}
    \<\simeq
    \commentaire{Évaluation de $log_{\frac{3}{2}}3$}
    \Theta\pars{n^{2.709511}}
\end{deriv}

\end{document}