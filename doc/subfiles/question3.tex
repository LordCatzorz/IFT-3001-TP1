% !TeX root = ../main.tex
\documentclass[class=article]{standalone}

\begin{document}
\centerline{\Huge \bf Question 3}
\section*{Algorithme 1}
\subsection*{Taille de l'instance}
La taille de l'instance est $n$, la valeur fournie en paramètre.

\subsection*{Opérateur de base}
Nous allons prendre comme opérateur de base les deux additions de c avec 1 ($c+1$).

Nous pouvons prendre deux opérateurs, puisqu'ils sont chacun appelé le plus souvent (à un constante près)
de chacun des deux boucles.

Le nombre de fois que ces opérateurs sont atteind dépend uniquement de $n$.

Le nombre de fois que les opérateurs de bases sonnt appeler peut être donné par la sommation suivante:

$C\pars{n} = \pars{\sum\limits_{i=1}^n \sum\limits_{j=i^3}^{n^3}1} + \pars{\sum\limits_{i=1}^{\floor{\sqrt{n}}} 1}$

Résolvons cette équation.

\begin{deriv}
    C\pars{n}
    \<=
    \commentaire{Définition de $C\pars{n}$}
    \pars{\sum\limits_{i=1}^n \sum\limits_{j=i^3}^{n^3}1} + \pars{\sum\limits_{i=1}^{\floor{\sqrt{n}}} 1}
    \<=
    \commentaire{Sommation d'une constante}
    \pars{\sum\limits_{i=1}^n \pars{\pars{n^3 - i^3 + 1} \cdot 1}} + \pars{\pars{\floor{\sqrt{n}} - 1 + 1} \cdot 1}
    \<=
    \commentaire{Simplification}
    \sum\limits_{i=1}^n \pars{n^3 - i^3 + 1} + \floor{\sqrt{n}}
\end{deriv}

Nous allons borner la sommation par une intégrale, et profiter de ces bornes pour évaluer le $\floor{\sqrt{n}}$

La sommation est non croissante.

Avec $f(i) = n^3 - i^3 + 1$ :

Résolvons $g(i) = \int\limits_1^{n+1} \pars{n^3 - x^3 + 1} dx$,
soit la borne inférieure de $\sum\limits_{i=1}^n \pars{n^3 - i^3 + 1}$.


\begin{deriv}
    g(i)
    \<=
    \commentaire{Définition de $g(i)$}
    \int\limits_1^{n+1} \pars{n^3 - x^3 + 1} dx
    \<=
    \commentaire{Distribution de l'intégration}
    \int\limits_1^{n+1} n^3 dx
    - \int\limits_1^{n+1} x^3 dx
    + \int\limits_1^{n+1} 1 dx
    \<=
    \commentaire{Intégration}
    \intr{1}{n+1}{n^3x}
    - \intr{1}{n+1}{\frac{x^4}{4}} dx
    + \intr{1}{n+1}{x} dx
    \<=
    \commentaire{Évaluation}
    \pars{n^3(n+1) - n^3(1)}
    - \pars{\frac{\pars{n+1}^4}{4} - \frac{1^4}{4}}
    + \pars{n+1 - 1}
    \<=
    \commentaire{Simplification}
    \pars{n^4 +n^3 - n^3}
    - \pars{\frac{\pars{n+1}^4-1}{4}}
    + n
    \<=
    \commentaire{Simplification}
    n^4 - \frac{\pars{n^2+2n+1}^2-1}{4} + n
    \<=
    \commentaire{Simplification}
    n^4 - \frac{n^4+4n^3+6n^2+4n+1-1}{4} + n
    \<=
    \commentaire{Simplification}
    n^4 - \frac{n^4+4n^3+6n^2+4n}{4} + n
    \<=
    \commentaire{Mettre sur la même base}
    \frac{4n^4}{4} - \frac{n^4+4n^3+6n^2+4n}{4} + \frac{4n}{4}
    \<=
    \commentaire{Rammener sur la même division}
    \frac{4n^4 - \pars{n^4+4n^3+6n^2+4n} + 4n}{4}
    \<=
    \commentaire{Distribuer la soustraction}
    \frac{4n^4 - n^4 - 4n^3 - 6n^2 - 4n + 4n}{4}
    \<=
    \commentaire{Simplifier}
    \frac{3n^4 - 4n^3 - 6n^2}{4}
\end{deriv}

Résolvons $h(i) = \int\limits_0^{n} \pars{n^3 - x^3 + 1} dx$,
soit la borne supérieure de $\sum\limits_{i=1}^n \pars{n^3 - i^3 + 1}$.

\begin{deriv}
    h(i)
    \<=
    \commentaire{Définition de $h(i)$}
    \int\limits_0^{n} \pars{n^3 - x^3 + 1} dx
    \<=
    \commentaire{Distribution de l'intégration}
    \int\limits_0^{n} n^3 dx
    - \int\limits_0^{n} x^3 dx
    + \int\limits_0^{n} 1 dx
    \<=
    \commentaire{Intégration}
    \intr{0}{n}{n^3x}
    - \intr{0}{n}{\frac{x^4}{4}} dx
    + \intr{0}{n}{x} dx
    \<=
    \commentaire{Évaluation}
    \pars{n^3(n) - n^3(0)}
    - \pars{\frac{\pars{n}^4}{4} - \frac{0^4}{4}}
    + \pars{n - 0}
    \<=
    \commentaire{Simplification}
    n^4 - \frac{n^4}{4} + n
    \<=
    \commentaire{Mettre sur la même base}
    \frac{4n^4}{4} - \frac{n^4}{4} + \frac{n}{4}
    \<=
    \commentaire{Rammener sur la même division}
    \frac{4n^4 - n^4 + 4n}{4}
    \<=
    \commentaire{Simplifier}
    \frac{3n^4 + 4n}{4}
\end{deriv}




\end{document}