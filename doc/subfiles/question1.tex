% !TeX root = ../main.tex
\documentclass[class=article]{standalone}
\begin{document}
\centerline{\Huge \bf Question 1}
\bigskip
\noindent{\bf Description}

L'algorithme se base sur ces observations suivantes:
\begin{enumerate}
    \item Zéro est l'élément absorbant de la multiplication;
    \item Le résultat est un produit d'éléments de vecteur (une multiplication);
    \item Lorsque deux zéro sont dans le vecteur, le produit sera toujours zéro;
    \item Nous pouvons émuler la non-production d'un élément, en divisant le produit par celui-ci, pour vu qu'il ne soit pas zéro.
\end{enumerate}

L'algorithme débute donc initialisant une variable pour conserver produit total (initialiser à 1), 
aisi qu'un vecteur qui conservera les indices des éléments absorbant (0).

Puis, il parcourt tous les éléments de notre vecteur entrant. Si l'élément est zéro, il ajoute l'index
de cet élément dans un nouveau vecteur. Sinon, il multiplie la variable par lui-même.

Cette boucle terminée, il entre dans une des trois embranchements dépendament de la taille du vecteur des indices éléments absorbants.

Premièrement, si le vecteur contient plus de un indice, alors nous savons que tous les éléments du vecteur résultat seront zéro, 
puisque nous sommes garantis qu'il y aura une multiplication par zéro pour obtenir le résultat.

Deuxièmement, si le vecteur contient un et un seul indice, alors tous les éléments du vecteur résultat seront
à zéro, à l'exception de l'élément à cet indice qui aura le résultat total du produit.

Finalement, si le vecteur ne contient aucun indice, alors chaque élément du vecteur résultat sera égale au produit, diviser par 
l'élément à l'indice correspodant du vecteur entrant.


\end{document} 