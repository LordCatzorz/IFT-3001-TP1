% !TeX root = ../main.tex
\documentclass[class=article]{standalone}
\begin{document}
\centerline{\Huge \bf Question 1}
\bigskip
\noindent{\bf Description textuelle}

L'algorithme se base sur ces observations suivantes:
\begin{enumerate}
    \item Zéro est l'élément absorbant de la multiplication;
    \item Le résultat est un produit d'éléments de vecteur (une multiplication);
    \item Lorsque deux zéro sont dans le vecteur, le produit sera toujours zéro;
    \item Nous pouvons émuler la non-production d'un élément, en divisant le produit par celui-ci, pour vu qu'il ne soit pas zéro.
\end{enumerate}

L'algorithme débute donc en initialisant une variable pour conserver le produit total (initialiser à 1), 
ainsi que deux variables "indicatices" (flags). Un pour conserver l'indice d'un élément absorbant et l'autre pour indiquer s'il y a plus d'un élément absorbant.

Puis, l'algorithme parcourt tous les éléments de notre vecteur entrant. S'il trouve pour la première fois un zéro, il affecter l'indice
où cet élément se trouve dans le vecteur. S'il en trouve un deuxième il affecte la valeur bouléen indiquant plusieurs zéros à vrai. 

Cette boucle terminée, il entre dans une des trois embranchements dépendanments de la valeur des variables indicatrices.

Premièrement, si la variable booléenne indiquant que plusieurs zéro ont été trouvés dans le vecteur entrant,
alors nous initialisons le vecteur résultat pour qu'il ne contienne que des zéros, puisque nous sommes garantis qu'il 
y aura une mutiplication par zéro pour tout ces éléments. 

Sinon, si la variable indiquant l'indice d'un zéro a été définie, alors tous les éléments du vecteur résultat seront
à zéro, à l'exception de l'élément à cet indice qui aura le résultat total du produit.

Finalement, si aucunne des deux variables indicatrices n'a été définie, alors chaque élément du vecteur résultat sera égale au produit, diviser par 
l'élément à l'indice correspondant du vecteur entrant.

\bigskip
\noindent{\bf Analyse de l'algorithme}

\end{document} 