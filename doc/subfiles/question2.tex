% !TeX root = ../main.tex
\documentclass[class=article]{standalone}

\begin{document}
\centerline{\Huge \bf Question 2}
\bigskip


\[
  C(n) =
  \begin{cases}
    0 & \text{si } n \leq 1 \\
    2 \cdot C(\floor{\frac{n}{3}}) + n & \text{si } n > 1 \\
  \end{cases}
\]

\section*{Résolution récurrence}
Nous allons commencer par essayer de résoudre cette récurrence, que pour les 
occurrences de $n = 3^k$ $\forall k \in \mathbb{N}$. Donc $k = log_3 n$

Nous aurons alors 

$C\pars{3^k} =
2 \cdot C\pars{\floor{\frac{3^k}{3}}} + 3^k =
2 \cdot C\pars{3^{k-1}}+3^k$

avec comme valeur de base

$C(3^0) = 0$

Résolvons:
\begin{deriv}
  C\pars{n}
  \<=
  \commentaire{Définition de $3^k$}
  C\pars{3^k}
  \<=
  \commentaire{$1^{ere}$ induction}
  2 \cdot C\pars{3^{k-1}}+3^k
  \<=
  \commentaire{$2^e$ induction}
  2 \cdot \crochs{2 \cdot C\pars{3^{k-2}}+3^{k-1}} +3^k
  \<=
  \commentaire{Simplification}
  2^2 \cdot C\pars{3^{k-2}} + 2 \cdot 3^{k-1} + 3^k
  \<=
  \commentaire{$3^e$ induction}
  2^2 \cdot \crochs{2 \cdot C\pars{3^{k-3}} + 3^{k-2}} + 2 \cdot 3^{k-1} + 3^k
  \<=
  \commentaire{Simplification}
  2^3 \cdot C\pars{3^{k-3}} + 2^{2} \cdot 3^{k-2} + 2 \cdot 3^{k-1} + 3^k
  \<=
  \commentaire{Suite}
  ...
  \<=
  \commentaire{$i^e$ induction}
  2^i \cdot C\pars{3^{k-i}} + 2^{i-1} \cdot 3^{k-(i-1)} + ... + 2^{2} \cdot 3^{k-2} +  2 \cdot  3^{k-1} + 3^k
  \<=
  \commentaire{Suite}
  ...
  \<=
  \commentaire{$k^e$ induction}
  2^k \cdot C\pars{3^{k-k}} + 2^{k-1} \cdot 3^{k-(k-1)} + ... + 2^{2} \cdot 3^{k-2} +  2 \cdot  3^{k-1} + 3^k
  \<=
  \commentaire{Simplification}
  2^k \cdot C\pars{3^{0}} + 2^{k-1} \cdot 3^{1} + ... + 2^{2} \cdot 3^{k-2} +  2^{1} \cdot  3^{k-1} + 2^{0} \cdot 3^k
  \<=
  \commentaire{Simplification}
  2^k \cdot C\pars{3^{0}} + \sum\limits_{i=1}^{k}\pars{2^{k-i} \cdot 3^i}
  \<=
  \commentaire{Simplification}
  2^k \cdot C\pars{1} + \sum\limits_{i=1}^{k}\pars{2^{k-i} \cdot 3^i}
  \<=
  \commentaire{Valeur de base}
  2^k \cdot 0 + \sum\limits_{i=1}^{k}\pars{2^{k-i} \cdot 3^i}
  \<=
  \commentaire{Simplification}
  \sum\limits_{i=1}^{k}\pars{2^{k-i} \cdot 3^i}
  \<=
  \commentaire{Règle expostant}
  \sum\limits_{i=1}^{k}\pars{\frac{2^{k}}{2^{i}} \cdot 3^i}
  \<=
  \commentaire{Séparer fraction}
  \sum\limits_{i=1}^{k}\pars{2^{k} \cdot \frac{1}{2^{i}} \cdot 3^i}
  \<=
  \commentaire{Simplification}
  \sum\limits_{i=1}^{k}\pars{2^{k} \cdot \frac{3^i}{2^{i}}}
  \<=
  \commentaire{Simplification}
  \sum\limits_{i=1}^{k}\pars{2^{k} \cdot \pars{\frac{3}{2}}^i}
  \<=
  \commentaire{Extraire produit constant de la sommation}
  2^{k} \cdot \sum\limits_{i=1}^{k}\pars{\pars{\frac{3}{2}}^i}
  \<=
  \commentaire{Ajuster borne de la sommation}
  2^{k} \cdot \pars{\sum\limits_{i=0}^{k}\pars{\pars{\frac{3}{2}}^i} - \pars{\frac{3}{2}}^0}
  \<=
  \commentaire{Simplification}
  2^{k} \cdot \pars{\sum\limits_{i=0}^{k}\pars{\pars{\frac{3}{2}}^i} - 1}
  \<=
  \commentaire{Application de la règle de sommation}
  2^{k} \cdot \pars{\frac{\pars{\frac{3}{2}}^{k+1}-1}{\frac{3}{2} - 1} - 1}
  \<=
  \commentaire{Simplification}
  2^{k} \cdot \pars{\frac{\pars{\frac{3}{2}}^{k+1}-1}{\frac{1}{2}} - 1}
  \<=
  \commentaire{Simplification}
  2^{k} \cdot \pars{2\cdot\pars{\pars{\frac{3}{2}}^{k+1}-1} - 1}
  \<=
  \commentaire{Simplification}
  2^{k} \cdot \pars{2\cdot{\pars{\frac{3}{2}}^{k+1}-2} - 1}
  \<=
  \commentaire{Simplification}
  2^{k} \cdot \pars{2\cdot{\pars{\frac{3}{2}}^{k+1}-3}}
  \<=
  \commentaire{Extraire un $\frac{3}{2}$}
  2^{k} \cdot \pars{2\cdot \frac{3}{2} \cdot {\pars{\frac{3}{2}}^{k}-3}}
  \<=
  \commentaire{Simplification}
  2^{k} \cdot \pars{3 \cdot {\pars{\frac{3}{2}}^{k}-3}}
  \<=
  \commentaire{Appliquer $k$ sur $\frac{3}{2}$}
  2^{k} \cdot \pars{3 \cdot \frac{3^{k}}{2^{k}}-3}
  \<=
  \commentaire{Appliquer multiplication $2^k$}
  2^{k} \cdot 3 \cdot \frac{3^{k}}{2^{k}}-2^{k} \cdot 3
  \<=
  \commentaire{Simplification}
  3 \cdot 3^{k}-2^{k} \cdot 3
  \<=
  \commentaire{Définition de $n$ et de $k$}
  3n-2^{log_3(n)} \cdot 3
  \<=
  \commentaire{Simplification}
  3 \pars{n-2^{log_3(n)}}
  \<=
  \commentaire{Règle des log}
  3 \pars{n-2^{log_3(2)log_2(n)}}
  \<=
  \commentaire{Règle des exposants}
  3 \pars{n-\pars{2^{log_2(n)}}^{log_3(2)}}
  \<=
  \commentaire{Simplification}
  3 \pars{n-n^{log_3(2)}}
  \<\approx
  \commentaire{Approximation du $log_3 2$}
  3 \pars{n-n^{0.63093}}
\end{deriv}

\section*{Notation asymptotique}
Nous pouvons trouver la notation asymptotique:

Borne supérieure:
\begin{deriv}
  3 \pars{n-n^{log_3(2)}}
  \<=
  \commentaire{Réécriture}
  3n-3n^{log_3(2)}
  \<\leq
  \commentaire{En ajoutant $3n^{log_3(2)}$}
  3n-3n^{log_3(2)}+3n^{log_3(2)}
  \<=
  \commentaire{Simplification}
  3n
  \<\in 
  \commentaire{Définition du $\BigO$ avec $c_2 = 3$ et $n_0 = 0$}
  \BigO\pars{n}
\end{deriv}

Borne inférieure:
\begin{deriv}
  3 \pars{n-n^{log_3(2)}}
  \<=
  \commentaire{Réécriture}
  3n-3n^{log_3(2)}
  \<\geq
  \commentaire{En retirant $n - 3n^{log_3(2)}$. Vrai $\forall n \geq 20$}
  3n-3n^{log_3(2)} - \pars{n - 3n^{log_3(2)}}
  \<=
  \commentaire{Appliquer la négation sur la parenthèse}
  3n-3n^{log_3(2)} - n + 3n^{log_3(2)}
  \<=
  \commentaire{Appliquer la négation}
  2n
  \<\in
  \commentaire{Définition de $\Omega$ avec $c_1 = 2$ et $n_0 = 20$}
  \Omega\pars{n}
\end{deriv}

Donc comme $3 \pars{n-n^{log_3(2)}} \in \Omega \pars {n}$ et $3 \pars{n-n^{log_3(2)}} \in \BigO \pars{n}$,
nous pouvons conclure que $3 \pars{n-n^{log_3(2)}} \in \Theta\pars{n}$.

Nous pouvons aussi conclure que $C\pars{n} \in \Theta\pars{n}$ pour $n = 3^k$ $\forall k \in \mathbb{N}$

\section*{Démonstration de la validité pour tout entier n}
Nous pouvons utiliser la règle de l'harmonie pour montrer que la notation asymptotique que
nous avons trouvé pour les $n = 3^k$ est valide $\forall n \in \mathbb{N}$. Pour ce faire, nous devons montrer 3 choses :

\begin{enumerate}
  \item $3 \pars{n-n^{log_3(2)}}$ doit être éventuellement non décroissante. 
  
  C'est le cas, en particulier lorsque $n \geq 1$.

  \item L'ordre de croissance, doit être une fonction harmonieuse, donc que $2n \in \Theta\pars{n}$.
  
  C'est le cas puisque 

  \begin{deriv}
    \lim\limits_{x\to\infty} \frac{2n}{n}
    \<=
    \commentaire{Simplification}
    \lim\limits_{x\to\infty} 2
    \<=
    \commentaire{Application de la limite}
    2 \commentaire{Valeur constante $> 0$, $2n \in \Theta\pars{n}$}
  \end{deriv} 

  \item On doit avoir $C\pars{n} \in \Theta\pars{n}$ pour $n = b^k$ $\forall k \in \mathbb{N}$
  
  C'est le cas puisque nous l'avons démontré dans la section "Notation asymptotique" avec 
  $b = 3$.
\end{enumerate}

Puisque nous répondons aux trois critères, nous pouvons conclure que $C(n) \in \Theta\pars{n}$ $\forall n \in \mathbb{N}$.

\end{document}