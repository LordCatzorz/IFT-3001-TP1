% !TeX root = ../main.tex
\documentclass[class=article]{standalone}
\usepackage{amsmath}
\usepackage{amsthm}
\usepackage{amssymb}

%%%%%%%%%%%%%%%%%%%%%%%%%%%%%%%%%%%%%%%%%%%%%%%%%%
% Environnement deriv pour les dérivations
%%%%%%%%%%%%%%%%%%%%%%%%%%%%%%%%%%%%%%%%%%%%%%%%%%

\newlength{\indentationFormule}
\setlength{\indentationFormule}{1.3em}
\newlength{\indentationTotaleFormule}
\newlength{\indentationCommentaire}
\setlength{\indentationCommentaire}{4em}
\newlength{\indentationDerivation}
\newlength{\largeurLangle}
\settowidth{\largeurLangle}{$\langle$\hspace*{.4em}}
\newlength{\largeurBoiteCommentaire}

\newenvironment{deriv}[1][\leftmargini]%
  {\setlength{\indentationDerivation}{#1}%
    \setlength{\indentationTotaleFormule}{\indentationFormule}
    \addtolength{\indentationTotaleFormule}{#1}
    \setlength{\largeurBoiteCommentaire}{\linewidth}
    \addtolength{\largeurBoiteCommentaire}{-\indentationFormule}
    \addtolength{\largeurBoiteCommentaire}{-\indentationCommentaire}
    \addtolength{\largeurBoiteCommentaire}{-\largeurLangle}
    \addtolength{\largeurBoiteCommentaire}{-\indentationDerivation}
    \begin{list}{}{\setlength{\leftmargin}{\indentationTotaleFormule}}
                   \setlength{\baselineskip}{1.3\baselineskip}
                   \item$}%
   {\hbox{}$\end{list}}  % \hbox{} évite des underfull lors d'usages
                         % sans \< et sans \commentaire (et avec
                         % des \\ pour changer de ligne).

\newcommand{\<}[1]{\\\hspace*{-\indentationFormule}\makebox(0,0)[bl]{$#1$}\hspace*{\indentationFormule}}

\newcommand{\commentaire}[1]{\hspace*{\indentationCommentaire}\langle\hspace*{.4em}%
    \begin{minipage}[t]{\largeurBoiteCommentaire}%
        #1 $\rangle$\\\hbox{}% Bizarre. Pourquoi ça prend \\ ici et aussi à la ligne suivante ?
                           % Sans \hbox{}, il y a des underfull.
    \end{minipage}\\[-1ex] % Sans ces deux \\, l'espace après un commentaire n'est pas le même quand
                           % le commentaire a plus d'une ligne.
}

\newcommand{\commentaireligne}[1]{$\mm ---#1$}{}
\newcommand{\sepcom}{\hspace{.4em}\&\hspace{.4em}\,}
%% Autre
\newcommand{\BigO}{\mathcal{O}}
\providecommand{\floor}[1]{\left \lfloor #1 \right \rfloor }
\providecommand{\pars}[1]{\left ( #1 \right ) }
\providecommand{\crochs}[1]{\left [ #1 \right ] }

\begin{document}
\centerline{\Huge \bf Question 2}
\bigskip


\[
  C(n) =
  \begin{cases}
    0 & \text{si } n \leq 1 \\
    2 \cdot C(\floor{\frac{n}{3}}) + n & \text{si } n > 1 \\
  \end{cases}
\]

Nous allons commencer par essayer de résoudre cette récurrence, que pour les 
occurances de $n = 3^k$ $\forall k \in \mathbb{N}$

Nous aurons alors 

$C\pars{3^k} =
2 \cdot C\pars{\floor{\frac{3^k}{3}}} + 3^k =
2 \cdot C\pars{3^{k-1}}+3^k$

avec comme valeur de base

$C(3^0) = 0$

Résolvons:
\begin{deriv}
    C\pars{3^k}
    \<=
    \commentaire{$1^{ere}$ induction}
    2 \cdot C\pars{3^{k-1}}+3^k
    \<=
    \commentaire{$2^e$ induction}
    2 \cdot \crochs{2 \cdot C\pars{3^{k-2}}+3^{k-1}} +3^k
    \<=
    \commentaire{Simplification}
    2^2 \cdot C\pars{3^{k-2}}+ 3^{k-1} + 3^k
    \<=
    \commentaire{$3^e$ induction}
    2^2 \cdot \crochs{2 \cdot C\pars{3^{k-3}}+3^{k-2}} + 3^{k-1} + 3^k
    \<=
    \commentaire{Simplification}
    2^3 \cdot C\pars{3^{k-3}} + 3^{k-2} + 3^{k-1} + 3^k
    \<=
    \commentaire{Suite}
    ...
    \<=
    \commentaire{$i^e$ induction}
    2^i \cdot C\pars{3^{k-i}} + 3^{k-(i-1)} + ... + 3^{k-2} + 3^{k-1} + 3^k
    \<=
    \commentaire{Suite}
    ...
    \<=
    \commentaire{$k^e$ induction}
    2^k \cdot C\pars{3^{k-k}} + \sum_{i=k-(k-1)}^{k} 3^i
    \<=
    \commentaire{Simplification}
    2^k \cdot C\pars{3^{0}} + \sum_{i=1}^{k} 3^i
    \<=
    \commentaire{Simplification}
    2^k \cdot C\pars{1} + \sum_{i=1}^{k} 3^i
    \<=
    \commentaire{Valeur de base}
    2^k \cdot 0 + \sum_{i=1}^{k} 3^i
    \<=
    \commentaire{Simplification}
    \sum_{i=1}^{k} 3^i
    \<=
    \commentaire{Ajout de valeurs neutres}
    \sum_{i=1}^{k} 3^i + 3^0 - 3^0
    \<=
    \commentaire{Simplification + Insérer addition dans sommation}
    \sum_{i=0}^{k} 3^i - 1
    \<=
    \commentaire{Règle de sommation}
    \frac{3^{k+1} - 1}{3 - 1} - 1
    \<=
    \commentaire{Simplification}
    \frac{3^{k+1} - 1}{2} - \frac{2}{2}
    \<=
    \commentaire{Simplification}
    \frac{3^{k+1} - 1 - 2}{2}
    \<=
    \commentaire{Simplification}
    \frac{3^{k+1} - 3}{2}
    \<=
    \commentaire{Extraction exposant}
    \frac{3 \cdot 3^k - 3}{2}
    \<=
    \commentaire{Définition de $3^k$}
    \frac{3n - 3}{2}
    \<=
    \commentaire{Réécriture}
    \frac{3}{2}n - \frac{3}{2}
\end{deriv}

\end{document}