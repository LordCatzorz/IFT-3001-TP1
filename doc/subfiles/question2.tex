% !TeX root = ../main.tex
\documentclass[class=article]{standalone}

\begin{document}
\centerline{\Huge \bf Question 2}
\bigskip


\[
  C(n) =
  \begin{cases}
    0 & \text{si } n \leq 1 \\
    2 \cdot C(\floor{\frac{n}{3}}) + n & \text{si } n > 1 \\
  \end{cases}
\]

\section*{Résolution récurence}
Nous allons commencer par essayer de résoudre cette récurrence, que pour les 
occurances de $n = 3^k$ $\forall k \in \mathbb{N}$

Nous aurons alors 

$C\pars{3^k} =
2 \cdot C\pars{\floor{\frac{3^k}{3}}} + 3^k =
2 \cdot C\pars{3^{k-1}}+3^k$

avec comme valeur de base

$C(3^0) = 0$

Résolvons:
\begin{deriv}
  C\pars{n}
  \<=
  \commentaire{Définition de $3^k$}
  C\pars{3^k}
  \<=
  \commentaire{$1^{ere}$ induction}
  2 \cdot C\pars{3^{k-1}}+3^k
  \<=
  \commentaire{$2^e$ induction}
  2 \cdot \crochs{2 \cdot C\pars{3^{k-2}}+3^{k-1}} +3^k
  \<=
  \commentaire{Simplification}
  2^2 \cdot C\pars{3^{k-2}}+ 3^{k-1} + 3^k
  \<=
  \commentaire{$3^e$ induction}
  2^2 \cdot \crochs{2 \cdot C\pars{3^{k-3}}+3^{k-2}} + 3^{k-1} + 3^k
  \<=
  \commentaire{Simplification}
  2^3 \cdot C\pars{3^{k-3}} + 3^{k-2} + 3^{k-1} + 3^k
  \<=
  \commentaire{Suite}
  ...
  \<=
  \commentaire{$i^e$ induction}
  2^i \cdot C\pars{3^{k-i}} + 3^{k-(i-1)} + ... + 3^{k-2} + 3^{k-1} + 3^k
  \<=
  \commentaire{Suite}
  ...
  \<=
  \commentaire{$k^e$ induction}
  2^k \cdot C\pars{3^{k-k}} + \sum\limits_{i=k-(k-1)}^{k} 3^i
  \<=
  \commentaire{Simplification}
  2^k \cdot C\pars{3^{0}} + \sum\limits_{i=1}^{k} 3^i
  \<=
  \commentaire{Simplification}
  2^k \cdot C\pars{1} + \sum\limits_{i=1}^{k} 3^i
  \<=
  \commentaire{Valeur de base}
  2^k \cdot 0 + \sum\limits_{i=1}^{k} 3^i
  \<=
  \commentaire{Simplification}
  \sum\limits_{i=1}^{k} 3^i
  \<=
  \commentaire{Ajout de valeurs neutres}
  \sum\limits_{i=1}^{k} 3^i + 3^0 - 3^0
  \<=
  \commentaire{Simplification + Insérer addition dans sommation}
  \sum\limits_{i=0}^{k} 3^i - 1
  \<=
  \commentaire{Règle de sommation}
  \frac{3^{k+1} - 1}{3 - 1} - 1
  \<=
  \commentaire{Simplification}
  \frac{3^{k+1} - 1}{2} - \frac{2}{2}
  \<=
  \commentaire{Simplification}
  \frac{3^{k+1} - 1 - 2}{2}
  \<=
  \commentaire{Simplification}
  \frac{3^{k+1} - 3}{2}
  \<=
  \commentaire{Extraction exposant}
  \frac{3 \cdot 3^k - 3}{2}
  \<=
  \commentaire{Définition de $3^k$}
  \frac{3n - 3}{2}
  \<=
  \commentaire{Réécriture}
  \frac{3}{2}n - \frac{3}{2}
\end{deriv}

\section*{Notation asymptotique}
Nous pouvons trouver la notation asymptotique:

Borne Supérieure:
\begin{deriv}
  \frac{3}{2}n - \frac{3}{2}
  \<\leq
  \commentaire{En ajoutant $\frac{3}{2}$}
  \frac{3}{2}n
  \<\in 
  \commentaire{Définition du $\BigO$ avec $c_2 = \frac{3}{2}$ et $n_0 = 0$}
  \BigO\pars{n}
\end{deriv}

Borne inférieure:
\begin{deriv}
  \frac{3}{2}n - \frac{3}{2}
  \<=
  \commentaire{Réécriture}
  \frac{3n}{2} - \frac{3}{2}
  \<\geq
  \commentaire{En retirant $\frac{n}{2} - \frac{3}{2}$}
  \frac{3n}{2} - \frac{3}{2} - \pars{\frac{n}{2} - \frac{3}{2}}
  \<=
  \commentaire{Appliquer la négation sur la parenthèse}
  \frac{3n}{2} - \frac{3}{2} - \frac{n}{2} + \frac{3}{2}
  \<=
  \commentaire{Appliquer la négation}
  \frac{3n}{2} - \frac{n}{2}
  \<=
  \commentaire{$-\frac{3}{2} + \frac{3}{2} = 0$}
  \frac{3n - n}{2}
  \<=
  \commentaire{Fusionner la soustraction de même base}
  \frac{2n}{2}
  \<=
  \commentaire{Simplification}
  n
  \<\in
  \commentaire{Définition de $\Omega$ avec $c_1 = 1$ et $n_0 = 0$}
  \Omega\pars{n}
\end{deriv}

Donc comme $\frac{3}{2}n - \frac{3}{2} \in \Omega \pars {n}$ et $\frac{3}{2}n - \frac{3}{2} \in \BigO \pars{n}$,
nous pouvons conclure que $\frac{3}{2}n - \frac{3}{2} \in \Theta\pars{n}$.

Nous pouvons aussi conclure que $C\pars{n} \in \Theta\pars{n}$ pour $n = 3^k$ $\forall k \in \mathbb{N}$

\section*{Démonstration de la validité pour tout entier n}
Nous pouvons utiliser la règles de l'harmonie pour montrer que la notation asymptotique que
nous avons trouver pour les $n = 3^k$ est valide $\forall n \in \mathbb{N}$. Pour ce faire, nous devons montrer 3 choses :

\begin{enumerate}
  \item $\frac{3}{2}n - \frac{3}{2}$ doit être éventuellement non décroissante. 
  
  C'est le cas puisque c'est une fonction linéaire avec une pente positive.

  \item L'ordre de croissance, doit être une fonction harmonieuse, donc que $2n \in \Theta\pars{n}$.
  
  C'est le cas puisque 

  \begin{deriv}
    \lim\limits_{x\to\infty} \frac{2n}{n}
    \<=
    \commentaire{Simplification}
    \lim\limits_{x\to\infty} 2
    \<=
    \commentaire{Application de la limit}
    2 \commentaire{Valeur constante $> 0$, $2n \in \Theta\pars{n}$}
  \end{deriv} 

  \item On doit avoir $C\pars{n} \in \Theta\pars{n}$ pour $n = b^k$ $\forall k \in \mathbb{N}$
  
  C'est le cas puisque nous l'avons démontré dans la section "Notation asymptotique" avec 
  $b = 3$.
\end{enumerate}

Puisque nous répondont au trois critères, nous pouvons conclure que $C(n) \in \Theta\pars{n}$ $\forall n \in \mathbb{N}$.

\end{document}